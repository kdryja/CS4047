%%%%%%%%%%%%%%%%%%%%%%%%%%%%%%%%%%%%%%%%%%%%%%%%%%%%%%%%%%%%%%%%%%%%%%%%%%%%%%%%
%2345678901234567890123456789012345678901234567890123456789012345678901234567890
%        1         2         3         4         5         6         7         8

\documentclass[letterpaper, 10 pt, conference]{ieeeconf}  % Comment this line out
                                                          % if you need a4paper
%\documentclass[a4paper, 10pt, conference]{ieeeconf}      % Use this line for a4
                                                          % paper

\IEEEoverridecommandlockouts                              % This command is only
                                                          % needed if you want to
                                                          % use the \thanks command
\overrideIEEEmargins
% See the \addtolength command later in the file to balance the column lengths
% on the last page of the document



% The following packages can be found on http:\\www.ctan.org
%\usepackage{graphics} % for pdf, bitmapped graphics files
%\usepackage{epsfig} % for postscript graphics files
%\usepackage{mathptmx} % assumes new font selection scheme installed
%\usepackage{times} % assumes new font selection scheme installed
%\usepackage{amsmath} % assumes amsmath package installed
%\usepackage{amssymb}  % assumes amsmath package installed
\usepackage{graphicx}

\title{\LARGE \bf
CS4047 In-Course Assessment
}

%\author{ \parbox{3 in}{\centering Huibert Kwakernaak*
%         \thanks{*Use the $\backslash$thanks command to put information here}\\
%         Faculty of Electrical Engineering, Mathematics and Computer Science\\
%         University of Twente\\
%         7500 AE Enschede, The Netherlands\\
%         {\tt\small h.kwakernaak@autsubmit.com}}
%         \hspace*{ 0.5 in}
%         \parbox{3 in}{ \centering Pradeep Misra**
%         \thanks{**The footnote marks may be inserted manually}\\
%        Department of Electrical Engineering \\
%         Wright State University\\
%         Dayton, OH 45435, USA\\
%         {\tt\small pmisra@cs.wright.edu}}
%}

\author{Konrad Dryja - 51552177 \\
  University of Aberdeen \\
  \today% <-this % stops a space
}


\begin{document}



\maketitle
\thispagestyle{empty}
\pagestyle{empty}


%%%%%%%%%%%%%%%%%%%%%%%%%%%%%%%%%%%%%%%%%%%%%%%%%%%%%%%%%%%%%%%%%%%%%%%%%%%%%%%%
\begin{abstract}

This report illustrates the difference in performance between a simple lexer of the ac language written in C and another created with Lex, based on their execution time with the same stress example ac program.

\end{abstract}


%%%%%%%%%%%%%%%%%%%%%%%%%%%%%%%%%%%%%%%%%%%%%%%%%%%%%%%%%%%%%%%%%%%%%%%%%%%%%%%%
\section{INTRODUCTION 150 words}

Lorem ipsum

\section{PROPOSED TECHNOLOGIES}

\subsection{Artificial Immune System aim at 150 words}
To make the scanning of characters more optimal than with if and else, a switch-case statement is used to identify the terminals. Previously the comments were just ignored but a modification was added for it to print "COMMENT" in those cases to have the same process as the Lex one.

\subsubsection{Strengths aim at 150 words} 
Lorem ipsum

\subsubsection{Weaknesses aim at 150 words}
Lorem ipsum

\subsubsection{Applications aim at 150 words}
Lorem ipsum

\subsection{Artificial Neural Networks 150 words}
With Lex one just has to define the regular expression rules for the tokens and how they are to be processed, Lex later generates a complete scanner coded in C, transforming the regular expression definitions into an equivalent finite automatomaton.

\subsubsection{Strengths aim at 150 words}
Lorem ipsum

\subsubsection{Weaknesses aim at 150 words}
Lorem ipsum

\subsubsection{Applications aim at 150 words}
Lorem ipsum


\section{COMBINATIONS - aim at 150 words}

Lorem ipsum

\section{CONCLUSIONS - aim at 200 words}

Lorem ipsum

\addtolength{\textheight}{-12cm}   % This command serves to balance the column lengths
                                  % on the last page of the document manually. It shortens
                                  % the textheight of the last page by a suitable amount.
                                  % This command does not take effect until the next page
                                  % so it should come on the page before the last. Make
                                  % sure that you do not shorten the textheight too much.

%%%%%%%%%%%%%%%%%%%%%%%%%%%%%%%%%%%%%%%%%%%%%%%%%%%%%%%%%%%%%%%%%%%%%%%%%%%%%%%%



%%%%%%%%%%%%%%%%%%%%%%%%%%%%%%%%%%%%%%%%%%%%%%%%%%%%%%%%%%%%%%%%%%%%%%%%%%%%%%%%



%%%%%%%%%%%%%%%%%%%%%%%%%%%%%%%%%%%%%%%%%%%%%%%%%%%%%%%%%%%%%%%%%%%%%%%%%%%%%%%%


\begin{thebibliography}{99}

\bibitem{c1} C. N. Fischer, R. K. Cytron \& R. J. LeBlanc. ÒCrafting a CompilerÓ, 2nd ed.,Boston: Pearson Education, 2010.

\end{thebibliography}

\end{document}

