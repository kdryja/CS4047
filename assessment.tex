%%%%%%%%%%%%%%%%%%%%%%%%%%%%%%%%%%%%%%%%%%%%%%%%%%%%%%%%%%%%%%%%%%%%%%%%%%%%%%%%
%2345678901234567890123456789012345678901234567890123456789012345678901234567890
%        1         2         3         4         5         6         7         8

\documentclass[letterpaper, 10 pt, conference]{ieeeconf}  % Comment this line out
                                                          % if you need a4paper
%\documentclass[a4paper, 10pt, conference]{ieeeconf}      % Use this line for a4
                                                          % paper

\IEEEoverridecommandlockouts                              % This command is only
                                                          % needed if you want to
                                                          % use the \thanks command
\overrideIEEEmargins
% See the \addtolength command later in the file to balance the column lengths
% on the last page of the document



% The following packages can be found on http:\\www.ctan.org
%\usepackage{graphics} % for pdf, bitmapped graphics files
%\usepackage{epsfig} % for postscript graphics files
%\usepackage{mathptmx} % assumes new font selection scheme installed
%\usepackage{times} % assumes new font selection scheme installed
%\usepackage{amsmath} % assumes amsmath package installed
%\usepackage{amssymb}  % assumes amsmath package installed
\usepackage{graphicx}

\title{\LARGE \bf
CS4047: In-Course Assessment
}

%\author{ \parbox{3 in}{\centering Huibert Kwakernaak*
%         \thanks{*Use the $\backslash$thanks command to put information here}\\
%         Faculty of Electrical Engineering, Mathematics and Computer Science\\
%         University of Twente\\
%         7500 AE Enschede, The Netherlands\\
%         {\tt\small h.kwakernaak@autsubmit.com}}
%         \hspace*{ 0.5 in}
%         \parbox{3 in}{ \centering Pradeep Misra**
%         \thanks{**The footnote marks may be inserted manually}\\
%        Department of Electrical Engineering \\
%         Wright State University\\
%         Dayton, OH 45435, USA\\
%         {\tt\small pmisra@cs.wright.edu}}
%}

\author{Konrad Dryja - 51552177 \\
  University of Aberdeen \\
  \today% <-this % stops a space
}


\begin{document}



\maketitle
\thispagestyle{empty}
\pagestyle{empty}


%%%%%%%%%%%%%%%%%%%%%%%%%%%%%%%%%%%%%%%%%%%%%%%%%%%%%%%%%%%%%%%%%%%%%%%%%%%%%%%%
\begin{abstract}

Lorem Ipsum

\end{abstract}


%%%%%%%%%%%%%%%%%%%%%%%%%%%%%%%%%%%%%%%%%%%%%%%%%%%%%%%%%%%%%%%%%%%%%%%%%%%%%%%%
\section{INTRODUCTION}

Bio-inspired computing is an ever-expanding area within computer science attempting to apply concepts that we observe in nature and in living organisms to modern algorithms in order to improve their efficiency and accuracy. The combination of efforts by scientists from different departments - such as biology (Genetic Algorithms) or sociology (Particle Swarm Optimisation) - enabled current techniques to learn from their mistakes and adapt to the changing circumstances \& environment. I strongly believe that introduction of such methods into our company would help us tackle problems such as stock prediction \cite{gunasekaran2011evaluation} or improving our current Machine Learning efforts - with the ultimate target of increasing the profit potential. In this report, I would like to bring our attention to two of the most common approaches: Artificial Immune Systems and Artificial Neural Networks, where I will explain the potential application and how useful they could be for our business.

\section{ARTIFICIAL IMMUNE SYSTEM}

Artificial Immune System (AIS) draws directly from biology of humans' (and not only) immune systems - organism's first line of defence against unwanted cells or viruses. The exact details and biological explanation behind those concepts can get incredibly complex very quickly, so for the sake of brevity, we can distinguish two major actors: antibodies (released by B / T cells) and antigens (the viruses). The main goal of antibodies is to identify and destroy to antigens - most importantly, the fight is usually a collective effort of the entire system, rather than of the individual cells. Moreover, one of the crucial concepts that carries over from biology into computing application is "self" and "not-self". The immune system should be able to distinguish between bodies belonging to the organism ("self") and the ones that haven't been recognized and thus should be eliminated ("not-self").

\subsection{Applications}
One of the more beneficial examples to our business would certainly be stock prediction. Gunasekaran et al. \cite{gunasekaran2011evaluation} presented an experiment where AIS was deployed to measure the trends and fluctuations on Bombay stock exchange. By feeding data for the year of 2009, the scientists were able to predict with a high accuracy the SENSEX Index for the year of 2010 (Root Mean Square Error of 103.106). While constructing the model, various indicators were used, such as Money Flow Index or Simple Moving Average. The study was researching the ultimate error difference between application of AIS and ANN - the former having achieved better results.

\subsection{Strengths aim at 150 words} 
Lorem ipsum

\subsection{Weaknesses aim at 150 words}
Lorem ipsum


\section{Artificial Neural Networks 150 words}
Lorem ipsum

\subsection{Strengths aim at 150 words}
Lorem ipsum

\subsection{Weaknesses aim at 150 words}
Lorem ipsum

\subsection{Applications aim at 150 words}
Lorem ipsum


\section{COMBINATIONS - aim at 150 words}

Lorem ipsum

\section{CONCLUSIONS - aim at 200 words}

Lorem ipsum\cite{gunasekaran2011evaluation}

\addtolength{\textheight}{-12cm}   % This command serves to balance the column lengths
                                  % on the last page of the document manually. It shortens
                                  % the textheight of the last page by a suitable amount.
                                  % This command does not take effect until the next page
                                  % so it should come on the page before the last. Make
                                  % sure that you do not shorten the textheight too much.

%%%%%%%%%%%%%%%%%%%%%%%%%%%%%%%%%%%%%%%%%%%%%%%%%%%%%%%%%%%%%%%%%%%%%%%%%%%%%%%%



%%%%%%%%%%%%%%%%%%%%%%%%%%%%%%%%%%%%%%%%%%%%%%%%%%%%%%%%%%%%%%%%%%%%%%%%%%%%%%%%



%%%%%%%%%%%%%%%%%%%%%%%%%%%%%%%%%%%%%%%%%%%%%%%%%%%%%%%%%%%%%%%%%%%%%%%%%%%%%%%%

\bibliographystyle{unsrt}
\bibliography{assessment}

\end{document}

